\chapter{Variation Under Domestication}
\minitoc

Causes of Variability -- Effects of Habit and the use and disuse
of Parts -- Correlated Variation -- Inheritance -- Character of
Domestic Varieties -- Difficulty of distinguishing between Varieties
and Species -- Origin of Domestic Varieties from one or more Species
-- Domestic Pigeons, their Differences and Origin -- Principles of
Selection, anciently followed, their Effects -- Methodical and
Unconscious Selection -- Unknown Origin of our Domestic Productions
-- Circumstances favourable to Man's power of Selection.


\section{Causes of Variability}
When we compare the individuals of the same variety or
sub-variety of our older cultivated plants and animals, one of the
first points which strikes us is, that they generally differ more
from each other than do the individuals of any one species or
variety in a state of nature. And if we reflect on the vast
diversity of the plants and animals which have been cultivated, and
which have varied during all ages under the most different climates
and treatment, we are driven to conclude that this great
variability is due to our domestic productions having been raised
under conditions of life not so uniform as, and somewhat different
from, those to which the parent species had been exposed under
nature. There is, also, some probability in the view propounded by
Andrew Knight, that this variability may be partly connected with
excess of food. It seems clear that organic beings must be exposed
during several generations to new conditions to cause any great
amount of variation; and that, when the organisation has once begun
to vary, it generally continues varying for many generations. No
case is on record of a variable organism ceasing to vary under
cultivation. Our oldest cultivated plants, such as wheat, still
yield new varieties: our oldest domesticated animals are still
capable of rapid improvement or modification.

As far as I am able to judge, after long attending to the
subject, the conditions of life appear to act in two ways—directly
on the whole organisation or on certain parts alone and in directly
by affecting the reproductive system. With respect to the direct
action, we must bear in mind that in every case, as Professor
Weismann has lately insisted, and as I have incidently shown in my
work on "Variation under Domestication," there are two factors:
namely, the nature of the organism and the nature of the
conditions. The former seems to be much the more important; for
nearly similar variations sometimes arise under, as far as we can
judge, dissimilar conditions; and, on the other hand, dissimilar
variations arise under conditions which appear to be nearly
uniform. The effects on the offspring are either definite or in
definite. They may be considered as definite when all or nearly all
the offspring of individuals exposed to certain conditions during
several generations are modified in the same manner. It is
extremely difficult to come to any conclusion in regard to the
extent of the changes which have been thus definitely induced.
There can, however, be little doubt about many slight changes, such
as size from the amount of food, colour from the nature of the
food, thickness of the skin and hair from climate, etc. Each of the
endless variations which we see in the plumage of our fowls must
have had some efficient cause; and if the same cause were to act
uniformly during a long series of generations on many individuals,
all probably would be modified in the same manner. Such facts as
the complex and extraordinary out growths which variably follow
from the insertion of a minute drop of poison by a gall-producing
insect, shows us what singular modifications might result in the
case of plants from a chemical change in the nature of the sap.

In definite variability is a much more common result of changed
conditions than definite variability, and has probably played a
more important part in the formation of our domestic races. We see
in definite variability in the endless slight peculiarities which
distinguish the individuals of the same species, and which cannot
be accounted for by inheritance from either parent or from some
more remote ancestor. Even strongly-marked differences occasionally
appear in the young of the same litter, and in seedlings from the
same seed-capsule. At long intervals of time, out of millions of
individuals reared in the same country and fed on nearly the same
food, deviations of structure so strongly pronounced as to deserve
to be called monstrosities arise; but monstrosities cannot be
separated by any distinct line from slighter variations. All such
changes of structure, whether extremely slight or strongly marked,
which appear among many individuals living together, may be
considered as the in definite effects of the conditions of life on
each individual organism, in nearly the same manner as the chill
effects different men in an in definite manner, according to their
state of body or constitution, causing coughs or colds, rheumatism,
or inflammation of various organs.

With respect to what I have called the in direct action of
changed conditions, namely, through the reproductive system of
being affected, we may infer that variability is thus induced,
partly from the fact of this system being extremely sensitive to
any change in the conditions, and partly from the similarity, as
Kolreuter and others have remarked, between the variability which
follows from the crossing of distinct species, and that which may
be observed with plants and animals when reared under new or
unnatural conditions. Many facts clearly show how eminently
susceptible the reproductive system is to very slight changes in
the surrounding conditions. Nothing is more easy than to tame an
animal, and few things more difficult than to get it to breed
freely under confinement, even when the male and female unite. How
many animals there are which will not breed, though kept in an
almost free state in their native country! This is generally, but
erroneously attributed to vitiated instincts. Many cultivated
plants display the utmost vigour, and yet rarely or never seed! In
some few cases it has been discovered that a very trifling change,
such as a little more or less water at some particular period of
growth, will determine whether or not a plant will produce seeds. I
cannot here give the details which I have collected and elsewhere
published on this curious subject; but to show how singular the
laws are which determine the reproduction of animals under
confinement, I may mention that carnivorous animals, even from the
tropics, breed in this country pretty freely under confinement,
with the exception of the plantigrades or bear family, which seldom
produce young; whereas, carnivorous birds, with the rarest
exception, hardly ever lay fertile eggs. Many exotic plants have
pollen utterly worthless, in the same condition as in the most
sterile hybrids. When, on the one hand, we see domesticated animals
and plants, though often weak and sickly, breeding freely under
confinement; and when, on the other hand, we see individuals,
though taken young from a state of nature perfectly tamed,
long-lived, and healthy (of which I could give numerous instances),
yet having their reproductive system so seriously affected by
unperceived causes as to fail to act, we need not be surprised at
this system, when it does act under confinement, acting
irregularly, and producing offspring somewhat unlike their parents.
I may add that as some organisms breed freely under the most
unnatural conditions—for instance, rabbits and ferrets kept in
hutches—showing that their reproductive organs are not easily
affected; so will some animals and plants withstand domestication
or cultivation, and vary very slightly—perhaps hardly more than in
a state of nature.

Some naturalists have maintained that all variations are
connected with the act of sexual reproduction; but this is
certainly an error; for I have given in another work a long list of
"sporting plants;" as they are called by gardeners; that is, of
plants which have suddenly produced a single bud with a new and
sometimes widely different character from that of the other buds on
the same plant. These bud variations, as they may be named, can be
propagated by grafts, offsets, etc., and sometimes by seed. They
occur rarely under nature, but are far from rare under culture. As
a single bud out of many thousands produced year after year on the
same tree under uniform conditions, has been known suddenly to
assume a new character; and as buds on distinct trees, growing
under different conditions, have sometimes yielded nearly the same
variety—for instance, buds on peach- trees producing nectarines,
and buds on common roses producing moss-roses— we clearly see that
the nature of the conditions is of subordinate importance in
comparison with the nature of the organism in determining each
particular form of variation; perhaps of not more importance than
the nature of the spark, by which a mass of combustible matter is
ignited, has in determining the nature of the flames.




\section{Effects Of Habit And Of The Use Or Disuse Of Parts; Correlated Variation; Inheritance}
Changed habits produce an inherited effect as in the period of
the flowering of plants when transported from one climate to
another. With animals the increased use or disuse of parts has had
a more marked influence; thus I find in the domestic duck that the
bones of the wing weigh less and the bones of the leg more, in
proportion to the whole skeleton, than do the same bones in the
wild duck; and this change may be safely attributed to the domestic
duck flying much less, and walking more, than its wild parents. The
great and inherited development of the udders in cows and goats in
countries where they are habitually milked, in comparison with
these organs in other countries, is probably another instance of
the effects of use. Not one of our domestic animals can be named
which has not in some country drooping ears; and the view which has
been suggested that the drooping is due to disuse of the muscles of
the ear, from the animals being seldom much alarmed, seems
probable.

Many laws regulate variation, some few of which can be dimly
seen, and will hereafter be briefly discussed. I will here only
allude to what may be called correlated variation. Important
changes in the embryo or larva will probably entail changes in the
mature animal. In monstrosities, the correlations between quite
distinct parts are very curious; and many instances are given in
Isidore Geoffroy St. Hilaire's great work on this subject. Breeders
believe that long limbs are almost always accompanied by an
elongated head. Some instances of correlation are quite whimsical;
thus cats which are entirely white and have blue eyes are generally
deaf; but it has been lately stated by Mr. Tait that this is
confined to the males. Colour and constitutional peculiarities go
together, of which many remarkable cases could be given among
animals and plants. From facts collected by Heusinger, it appears
that white sheep and pigs are injured by certain plants, while
dark-coloured individuals escape: Professor Wyman has recently
communicated to me a good illustration of this fact; on asking some
farmers in Virginia how it was that all their pigs were black, they
informed him that the pigs ate the paint-root (Lachnanthes), which
coloured their bones pink, and which caused the hoofs of all but
the black varieties to drop off; and one of the "crackers" (i.e.
Virginia squatters) added, "we select the black members of a litter
for raising, as they alone have a good chance of living." Hairless
dogs have imperfect teeth; long-haired and coarse-haired animals
are apt to have, as is asserted, long or many horns; pigeons with
feathered feet have skin between their outer toes; pigeons with
short beaks have small feet, and those with long beaks large feet.
Hence if man goes on selecting, and thus augmenting, any
peculiarity, he will almost certainly modify unintentionally other
parts of the structure, owing to the mysterious laws of
correlation.

The results of the various, unknown, or but dimly understood
laws of variation are infinitely complex and diversified. It is
well worth while carefully to study the several treatises on some
of our old cultivated plants, as on the hyacinth, potato, even the
dahlia, etc.; and it is really surprising to note the endless
points of structure and constitution in which the varieties and
sub-varieties differ slightly from each other. The whole
organisation seems to have become plastic, and departs in a slight
degree from that of the parental type.

Any variation which is not inherited is unimportant for us. But
the number and diversity of inheritable deviations of structure,
both those of slight and those of considerable physiological
importance, are endless. Dr. Prosper Lucas' treatise, in two large
volumes, is the fullest and the best on this subject. No breeder
doubts how strong is the tendency to inheritance; that like
produces like is his fundamental belief: doubts have been thrown on
this principle only by theoretical writers. When any deviation of
structure often appears, and we see it in the father and child, we
cannot tell whether it may not be due to the same cause having
acted on both; but when among individuals, apparently exposed to
the same conditions, any very rare deviation, due to some
extraordinary combination of circumstances, appears in the
parent—say, once among several million individuals—and it reappears
in the child, the mere doctrine of chances almost compels us to
attribute its reappearance to inheritance. Every one must have
heard of cases of albinism, prickly skin, hairy bodies, etc.,
appearing in several members of the same family. If strange and
rare deviations of structure are truly inherited, less strange and
commoner deviations may be freely admitted to be inheritable.
Perhaps the correct way of viewing the whole subject would be, to
look at the inheritance of every character whatever as the rule,
and non-inheritance as the anomaly.

The laws governing inheritance are for the most part unknown; no
one can say why the same peculiarity in different individuals of
the same species, or in different species, is sometimes inherited
and sometimes not so; why the child often reverts in certain
characteristics to its grandfather or grandmother or more remote
ancestor; why a peculiarity is often transmitted from one sex to
both sexes, or to one sex alone, more commonly but not exclusively
to the like sex. It is a fact of some importance to us, that
peculiarities appearing in the males of our domestic breeds are
often transmitted, either exclusively or in a much greater degree,
to the males alone. A much more important rule, which I think may
be trusted, is that, at whatever period of life a peculiarity first
appears, it tends to reappear in the offspring at a corresponding
age, though sometimes earlier. In many cases this could not be
otherwise; thus the inherited peculiarities in the horns of cattle
could appear only in the offspring when nearly mature;
peculiarities in the silk-worm are known to appear at the
corresponding caterpillar or cocoon stage. But hereditary diseases
and some other facts make me believe that the rule has a wider
extension, and that, when there is no apparent reason why a
peculiarity should appear at any particular age, yet that it does
tend to appear in the offspring at the same period at which it
first appeared in the parent. I believe this rule to be of the
highest importance in explaining the laws of embryology. These
remarks are of course confined to the first APPEARANCE of the
peculiarity, and not to the primary cause which may have acted on
the ovules or on the male element; in nearly the same manner as the
increased length of the horns in the offspring from a short-horned
cow by a long-horned bull, though appearing late in life, is
clearly due to the male element.

Having alluded to the subject of reversion, I may here refer to
a statement often made by naturalists—namely, that our domestic
varieties, when run wild, gradually but invariably revert in
character to their aboriginal stocks. Hence it has been argued that
no deductions can be drawn from domestic races to species in a
state of nature. I have in vain endeavoured to discover on what
decisive facts the above statement has so often and so boldly been
made. There would be great difficulty in proving its truth: we may
safely conclude that very many of the most strongly marked domestic
varieties could not possibly live in a wild state. In many cases we
do not know what the aboriginal stock was, and so could not tell
whether or not nearly perfect reversion had ensued. It would be
necessary, in order to prevent the effects of intercrossing, that
only a single variety should be turned loose in its new home.
Nevertheless, as our varieties certainly do occasionally revert in
some of their characters to ancestral forms, it seems to me not
improbable that if we could succeed in naturalising, or were to
cultivate, during many generations, the several races, for
instance, of the cabbage, in very poor soil—in which case, however,
some effect would have to be attributed to the DEFINITE action of
the poor soil —that they would, to a large extent, or even wholly,
revert to the wild aboriginal stock. Whether or not the experiment
would succeed is not of great importance for our line of argument;
for by the experiment itself the conditions of life are changed. If
it could be shown that our domestic varieties manifested a strong
tendency to reversion—that is, to lose their acquired characters,
while kept under the same conditions and while kept in a
considerable body, so that free intercrossing might check, by
blending together, any slight deviations in their structure, in
such case, I grant that we could deduce nothing from domestic
varieties in regard to species. But there is not a shadow of
evidence in favour of this view: to assert that we could not breed
our cart and race-horses, long and short-horned cattle, and poultry
of various breeds, and esculent vegetables, for an unlimited number
of generations, would be opposed to all experience.


\section{Character Of Domestic Varieties; Difficulty Of Distinguishing Between Varieties And Species; Origin Of Domestic Varieties From One Or More Species}

When we look to the hereditary varieties or races of our
domestic animals and plants, and compare them with closely allied
species, we generally perceive in each domestic race, as already
remarked, less uniformity of character than in true species.
Domestic races often have a somewhat monstrous character; by which
I mean, that, although differing from each other and from other
species of the same genus, in several trifling respects, they often
differ in an extreme degree in some one part, both when compared
one with another, and more especially when compared with the
species under nature to which they are nearest allied. With these
exceptions (and with that of the perfect fertility of varieties
when crossed—a subject hereafter to be discussed), domestic races
of the same species differ from each other in the same manner as do
the closely allied species of the same genus in a state of nature,
but the differences in most cases are less in degree. This must be
admitted as true, for the domestic races of many animals and plants
have been ranked by some competent judges as the descendants of
aboriginally distinct species, and by other competent judges as
mere varieties. If any well marked distinction existed between a
domestic race and a species, this source of doubt would not so
perpetually recur. It has often been stated that domestic races do
not differ from each other in characters of generic value. It can
be shown that this statement is not correct; but naturalists differ
much in determining what characters are of generic value; all such
valuations being at present empirical. When it is explained how
genera originate under nature, it will be seen that we have no
right to expect often to find a generic amount of difference in our
domesticated races.

In attempting to estimate the amount of structural difference
between allied domestic races, we are soon involved in doubt, from
not knowing whether they are descended from one or several parent
species. This point, if it could be cleared up, would be
interesting; if, for instance, it could be shown that the
greyhound, bloodhound, terrier, spaniel and bull-dog, which we all
know propagate their kind truly, were the offspring of any single
species, then such facts would have great weight in making us doubt
about the immutability of the many closely allied natural
species—for instance, of the many foxes—inhabiting the different
quarters of the world. I do not believe, as we shall presently see,
that the whole amount of difference between the several breeds of
the dog has been produced under domestication; I believe that a
small part of the difference is due to their being descended from
distinct species. In the case of strongly marked races of some
other domesticated species, there is presumptive or even strong
evidence that all are descended from a single wild stock.</p>
<p>It has often been assumed that man has chosen for domestication
animals and plants having an extraordinary inherent tendency to
vary, and likewise to withstand diverse climates. I do not dispute
that these capacities have added largely to the value of most of
our domesticated productions; but how could a savage possibly know,
when he first tamed an animal, whether it would vary in succeeding
generations, and whether it would endure other climates? Has the
little variability of the ass and goose, or the small power of
endurance of warmth by the reindeer, or of cold by the common
camel, prevented their domestication? I cannot doubt that if other
animals and plants, equal in number to our domesticated
productions, and belonging to equally diverse classes and
countries, were taken from a state of nature, and could be made to
breed for an equal number of generations under domestication, they
would on an average vary as largely as the parent species of our
existing domesticated productions have varied.

In the case of most of our anciently domesticated animals and
plants, it is not possible to come to any definite conclusion,
whether they are descended from one or several wild species. The
argument mainly relied on by those who believe in the multiple
origin of our domestic animals is, that we find in the most ancient
times, on the monuments of Egypt, and in the lake- habitations of
Switzerland, much diversity in the breeds; and that some of these
ancient breeds closely resemble, or are even identical with, those
still existing. But this only throws far backward the history of
civilisation, and shows that animals were domesticated at a much
earlier period than has hitherto been supposed. The
lake-inhabitants of Switzerland cultivated several kinds of wheat
and barley, the pea, the poppy for oil and flax; and they possessed
several domesticated animals. They also carried on commerce with
other nations. All this clearly shows, as Heer has remarked, that
they had at this early age progressed considerably in civilisation;
and this again implies a long continued previous period of less
advanced civilisation, during which the domesticated animals, kept
by different tribes in different districts, might have varied and
given rise to distinct races. Since the discovery of flint tools in
the superficial formations of many parts of the world, all
geologists believe that barbarian men existed at an enormously
remote period; and we know that at the present day there is hardly
a tribe so barbarous as not to have domesticated at least the
dog.

The origin of most of our domestic animals will probably forever
remain vague. But I may here state that, looking to the domestic
dogs of the whole world, I have, after a laborious collection of
all known facts, come to the conclusion that several wild species
of Canidae have been tamed, and that their blood, in some cases
mingled together, flows in the veins of our domestic breeds. In
regard to sheep and goats I can form no decided opinion. From facts
communicated to me by Mr. Blyth, on the habits, voice, constitution
and structure of the humped Indian cattle, it is almost certain
that they are descended from a different aboriginal stock from our
European cattle; and some competent judges believe that these
latter have had two or three wild progenitors, whether or not these
deserve to be called species. This conclusion, as well as that of
the specific distinction between the humped and common cattle, may,
indeed, be looked upon as established by the admirable researches
of Professor Rutimeyer. With respect to horses, from reasons which
I cannot here give, I am doubtfully inclined to believe, in
opposition to several authors, that all the races belong to the
same species. Having kept nearly all the English breeds of the fowl
alive, having bred and crossed them, and examined their skeletons,
it appears to me almost certain that all are the descendants of the
wild Indian fowl, Gallus bankiva; and this is the conclusion of Mr.
Blyth, and of others who have studied this bird in India. In regard
to ducks and rabbits, some breeds of which differ much from each
other, the evidence is clear that they are all descended from the
common duck and wild rabbit.

The doctrine of the origin of our several domestic races from
several aboriginal stocks, has been carried to an absurd extreme by
some authors. They believe that every race which breeds true, let
the distinctive characters be ever so slight, has had its wild
prototype. At this rate there must have existed at least a score of
species of wild cattle, as many sheep, and several goats, in Europe
alone, and several even within Great Britain. One author believes
that there formerly existed eleven wild species of sheep peculiar
to Great Britain! When we bear in mind that Britain has now not one
peculiar mammal, and France but few distinct from those of Germany,
and so with Hungary, Spain, etc., but that each of these kingdoms
possesses several peculiar breeds of cattle, sheep, etc., we must
admit that many domestic breeds must have originated in Europe; for
whence otherwise could they have been derived? So it is in India.
Even in the case of the breeds of the domestic dog throughout the
world, which I admit are descended from several wild species, it
cannot be doubted that there has been an immense amount of
inherited variation; for who will believe that animals closely
resembling the Italian greyhound, the bloodhound, the bull-dog,
pug-dog, or Blenheim spaniel, etc.—so unlike all wild Canidae—ever
existed in a state of nature? It has often been loosely said that
all our races of dogs have been produced by the crossing of a few
aboriginal species; but by crossing we can only get forms in some
degree intermediate between their parents; and if we account for
our several domestic races by this process, we must admit the
former existence of the most extreme forms, as the Italian
greyhound, bloodhound, bull-dog, etc., in the wild state. Moreover,
the possibility of making distinct races by crossing has been
greatly exaggerated. Many cases are on record showing that a race
may be modified by occasional crosses if aided by the careful
selection of the individuals which present the desired character;
but to obtain a race intermediate between two quite distinct races
would be very difficult. Sir J. Sebright expressly experimented
with this object and failed. The offspring from the first cross
between two pure breeds is tolerably and sometimes (as I have found
with pigeons) quite uniform in character, and every thing seems
simple enough; but when these mongrels are crossed one with another
for several generations, hardly two of them are alike, and then the
difficulty of the task becomes manifest.